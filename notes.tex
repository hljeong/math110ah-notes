\documentclass{notes}

\class{MATH 110AH (Algebra)}

\notusingsubsection

\begin{document}

\section{9.23 Friday Week 0}

Groups were invented by Évariste Galois circa 1830 to discuss symmetries in mathematical objects.
For instance, the group $G$ of ``symmetries of equilateral triangles'' contains a total of $3 \cdot 2 = 6$ elements consisting of rotations and reflections.

\begin{thm}
  Finite simple groups are classified.
\end{thm}

\underline{\smash{{\boldmath \bfseries Proving things about the integers}}}

Try to use ``simple'' facts about the integers to prove complicated ones.
Recall the number systems: 
\begin{itemize}
  \item the integers $\mathbb Z = \left \{ \dots, -3, -2, -1, 0, 1, 2, 3, \dots \right \}$, 

  \item the rational numbers $\mathbb Q = \left \{ \frac{a}{b} : a, b \in \mathbb Z, b \neq 0 \right \}$, and 

  \item the real numbers $\mathbb R$ contains $\mathbb Q, \sqrt 2, \pi, \dots$, ``all the points on the line.''
\end{itemize}

\begin{defn}
  A {\boldmath \bfseries field $\mathbb F$} (e.g. $\mathbb Q, \mathbb R, \mathbb C$, not $\mathbb Z$) is a set with elements $0, 1 \in \mathbb F$, $0 \neq 1$, and operators $+$ (addition) and $\cdot$ (multiplication), where for all $x, y \in \mathbb F$, $x + y, x y \in \mathbb F$, such that
  \begin{enumerate}
    \item $+$ is associative, commutative, $0$ is its identity, and has inverses.
    That is, 
    \begin{itemize}
      \item $\forall \, x, y, z \in \mathbb F: (x + y) + z = x + (y + z)$, 

      \item $\forall \, x, y \in \mathbb F: x + y = y + x$, 

      \item $\forall \, x \in \mathbb F: 0 + x = x$, and

      \item $\forall \, x \in \mathbb F \; \exists \, y \in \mathbb F: x + y = 0$, and writing $y = -x$.
    \end{itemize}

    \item $\cdot$ is associative, commutative, $1$ is its identity, and nonzeros have inverse.
    That is, 
    \begin{itemize}
      \item $\forall \, x, y, z \in \mathbb F: (x y) z = x (y z)$, 

      \item $\forall \, x, y \in \mathbb F: x y = y x$, 

      \item $\forall \, x \in \mathbb F: 1 \cdot x = x$, and 

      \item $\forall \, x \in \mathbb F: x \neq 0 \Rightarrow \exists \, y \in \mathbb F: x y = 1$.
    \end{itemize}

    \item Distributive law: $\forall \, x, y, z \in \mathbb F: x (y + z) = x y + x z$.
  \end{enumerate}
\end{defn}

\begin{defn}
  An {\boldmath \bfseries ordered field} (for example, $\mathbb R, \mathbb Q$, not $\mathbb C$) $\mathbb F$ is a field with a given subset $P \subset \mathbb F$ called the positive elements such that 
  \begin{enumerate}
    \item for all $x \in \mathbb F$, exactly one of $x \in P$, $x = 0$, $-x \in P$ is true (we say $x \in \mathbb F$ is negative if $-x \in P$), and 

    \item for all $x, y \in P$, $x + y, x y \in P$.
  \end{enumerate}
\end{defn}

\newpage

\underline{\smash{\boldmath \bfseries How to use these axioms to prove inequalities}}

\begin{defn}
  For an ordered field $\mathbb F$, we say for $x, y \in \mathbb F$ that {\boldmath \bfseries $x < y$} if $y - x = y + (-x)$ is positive (so $x$ is positive iff $x > 0$).

  Likewise {\boldmath \bfseries $x \leq y$} if $y - x$ is positive or $0$.
\end{defn}

\begin{lem}
  \begin{enumerate}
    \item For any $x, y \in \mathbb R$, exactly one of $x < y$, $x = y$, $x > y$ is true.

    \item $1$ is positive.

    \item For $a, b, c \in \mathbb R$, if $a < b$ then $a + c < b + c$.

    \item If $a, b, c \in \mathbb R$, $a \geq 0$, and $b \geq c$, then $a b \geq a c$.
  \end{enumerate}
\end{lem}

\begin{prf}
  \begin{enumerate}
    \item $y - x$ is either positive, negative, or $0$.

    \item Note that $1 \neq 0$.
    Then either $1$ is positive or $-1$ is positive.
    If $-1$ is positive then $(-1)^2 = 1$ is positive, resulting in a contradiction.

    \item Note that $(b + c) - (a + c) = b - a$ is positive.

    \item Note that for all $a \in \mathbb R$ we have $0 \cdot a = 0$.
    Then 
    \begin{align*}
      a (b - c) &\geq 0 \\ 
      a b - a c &\geq 0 \\ 
      a b &\geq a c.
    \end{align*}
  \end{enumerate}
\end{prf}

\underline{\smash{{\boldmath \bfseries The big property of the integers is the inductive or well-ordering principle}}}

The integers $\mathbb Z$ are a subset of $\mathbb Q$ (or $\mathbb R$).
There are $0, 1 \in \mathbb Z$ and if $x, y \in \mathbb Z$ then $x + y, x y, -x \in \mathbb Z$.

\begin{thm}[Well-ordering principle]
  Let $\mathbb Z^+ = \left \{ x \in \mathbb Z : x > 0 \right \} = \left \{ 1, 2, 3, 4, \dots \right \}$.
  Let $S$ be a nonempty subset of $\mathbb Z^+$.
  Then $S$ contains a smallest element, that is, 
  \[
    \exists \, x \in S \; \forall \, y \in S: x \leq y.
  \]
\end{thm}

An example of what we can prove using this is: 
\begin{prop}
  There is no integer $N$ with $0 < N < 1$.
\end{prop}

\begin{prf}
  Let $S = \left \{ n \in \mathbb Z : 0 < n < 1 \right \}$.
  Let $S \neq \varnothing$, then by the well-ordering principle $S$ has a smallest element $N$.
  Since $N < 1$, $N^2 < N \cdot 1 = N$.
  Since $N^2$ is also an integer, this is a contradiction.
  Then $S = \varnothing$, that is, there is no integer $N \in (0, 1)$.
\end{prf}

\begin{thm}[Induction]
  For each $n \in \mathbb Z^+$, let $P(n)$ be a statement that could be true or false.
  Suppose $P(1)$ is true and that for any $n \in \mathbb Z^+$, if $P(n)$ is true then $P(n + 1)$ is true.
  Then $P(n)$ is true for all $n \in \mathbb Z^+$.
\end{thm}

\begin{prf}
  Let $S = \left \{ n \in \mathbb Z^+ : P(n) \text{ is false} \right \}$.
  Suppose $S \neq \varnothing$, then by the well-ordering principle $S$ has a smallest element $N$.
  Since $P(1)$ is true, $N \neq 1$.
  Then $N > 1$.
  Then $N \geq 2$ since there are no integers in $(1, 2)$.
  Then $N - 1 \in \mathbb Z^+$.
  Since $P(N)$ would be true if $P(N - 1)$ were true, $P(N - 1)$ is not true.
  Then $N - 1 \in S$, a contradiction.
  Then $S = \varnothing$.
\end{prf}

\newpage

\section{9.26 Monday Week 1}

\begin{lem}
  For every $x \in \mathbb R$, $0 \cdot x = 0$.
\end{lem}

\begin{prf}
  By the distributive law, $0 = 0 + 0$, so for any $x \in \mathbb R$, we have 
  \begin{align*}
    0 \cdot x &= (0 + 0) \cdot x \\ 
    0 \cdot x + (-(0 \cdot x)) &= (0 + 0) \cdot x + (-(0 \cdot x)) \\ 
    0 &= 0 \cdot x.
  \end{align*}
\end{prf}

\begin{lem}
  For every $x \in \mathbb R$, $(-1) \cdot x = -x$.
\end{lem}

\begin{prf}
  Note that $0 = 0 \cdot x = (1 + (-1)) \cdot x = 1 \cdot x + (-1) \cdot x = x + (-1) \cdot x$.
  Adding $-x$ to both sides, we have $-x = (-1) \cdot x$.
\end{prf}

\begin{lem}
  For any $x, y, z \in \mathbb R$ with $x \neq 0$, if $x y = x z$, then $y = z$.
\end{lem}

\begin{prf}
  We know, since $x \neq 0$, there exists a real number $\overline x$ such that $x \left ( \frac{1}{x} \right ) = 1$.
  Then if $x y = x z$, then $\left ( \frac{1}{x} \right ) x y = \left ( \frac{1}{x} \right ) x z$, so $1 \cdot y = 1 \cdot z$, that is, $y = z$.
\end{prf}

\underline{\smash{{\boldmath \bfseries Proving things about the integer, gcd, prime factorization}}}

\begin{defn}
  For integers $x$ and $y$, we say {\boldmath \bfseries $x \mid y$} or ``{\boldmath \bfseries $x$ divides $y$}'' or ``{\boldmath \bfseries $y$ is a multiple of $x$}'' if $\exists \, z \in \mathbb Z: x z = y$.
  That is, if $x \neq 0$, $\frac{y}{x}$ is an integer.
\end{defn}

\underline{\smash{{\boldmath \bfseries Remark}}} For any $x \in \mathbb Z$, $1 \mid x$.
Also, for any $x \in \mathbb Z$, $x \mid 0$ since $x \cdot 0 = 0$.

Also, if $x$ is a nonzero integer then any integer $m$ with $m \mid x$ has $\left | m \right | \leq \left | x \right |$.

\begin{eg}
  The integers dividing $10 = 2 \cdot 5$ are $-10, -5, -2, -1, 1, 2, 5, 10$.
\end{eg}

\underline{\smash{{\boldmath \bfseries Remark}}} One fact about $\mathbb R$ is the Archimedean property: for every $x \in \mathbb R$, there exists an integer $y$ with $x < y$.
It follows, by multiplying by $-1$, $\forall \, x \in \mathbb R \; \exists \, y \in \mathbb Z: y < x$.

\underline{\smash{{\boldmath \bfseries Notation}}} For any $x \in \mathbb R$, $\left \lfloor x \right \rfloor :=$ the largest integer $\leq x$ and $\left \lceil x \right \rceil :=$ the smallest integer $\geq x$.

This follows form well-ordering that a subset of $Z$, bounded below and not empty, has a smallest element.

\begin{defn}
  A subset $S \subset \mathbb R$ is {\boldmath \bfseries bounded below} if $\exists \, a \in \mathbb R \; \forall \, x \in S: x \geq a$.
\end{defn}

\begin{thm}[Division algorithm]
  Let $x$ be a positive integer and $y$ any integer.
  Then there are (unique) integers $q$ and $r$ such that $y = q x + r$, and $0 \leq r \leq x - 1$.
\end{thm}

\begin{prf}
  Let $q = \left \lceil \frac{y}{x} \right \rceil (\in \mathbb Z)$.
  Define $r = y - q x (\in \mathbb Z)$.
  Clearly, $y = q x + r$.
  Here $q \leq \frac{y}{x}$, so $q x \leq y$ (since $x > 0$), so $r \geq 0$.

  Also, $q + 1 > \frac{y}{x}$.
  So (since $x > 0$) $q x + x > y$, so $r = y - q x < x$.
  Since $r \in \mathbb Z$, $r \leq x - 1$.
\end{prf}

\begin{defn}
  A positive integer $p$ is {\boldmath \bfseries prime} if $p > 1$ and the only positive integers dividing $p$ are $1$ and $p$.
\end{defn}

\begin{defn}
  For integers $x$ and $y$ not both $0$, the {\boldmath \bfseries greatest common divisor $= \gcd(x, y)$} is the largest integer that divides $x$ and $y$.
\end{defn}

That makes sense because $1 \mid x$ and $1 \mid y$, and (if $y \neq 0$), any integer dividing $y$ is $\leq \left | y \right |$.

\begin{thm}[Euclid, 300 BCE]
  For any integers $x, y$, not both $0$, there are integers $m, n$ with $\gcd(x, y) = m x + n y$.
\end{thm}

\begin{prf}
  The hypothesis and conclusion do not change if $x$ or $y$ is multiplied by $-1$.
  Assume $x, y \geq 0$.
  By switching $x$ and $y$ if needed, assume $0 \leq x \leq y$ and $y > 0$ since they are not both $0$.

  We prove this by induction on $y$.

  For $y = 1$, we have $x = 0$ or $x = 1$, and the conclusion is true: $\gcd(0, 1) = 1 = 0 \cdot 0 + 1 \cdot 1$ and $\gcd(1, 1) = 1 = 0 \cdot 1 + 1 \cdot 1$.

  Suppose now that $y \geq 2$ and the result holds for smaller $y$'s.
  If $x = 0$ then $\gcd(0, y) = y = 0 \cdot 0 + 1 \cdot y$.
  If $x = y$ then $\gcd(x, y) = y = 0 \cdot x + 1 \cdot y$.

  Now assume $0 < x < y$.
  Then the division algorithm gives $y = q x + r$ where $q, r \in \mathbb Z$ and $0 \leq r \leq x - 1$.
  Then $\gcd(x, y) = \gcd(r, x)$ because an integer divides both $x$ and $y$ iff it divides $r = y - q x$.
  Using induction, $\gcd(x, y) = \gcd(r, x) = m r + n x$ for some $m, n \in \mathbb Z$.
  Then $\gcd(x, y) = m (y - q x) + n x = (n - m q) x + m y$.

  Then induction is complete.
\end{prf}

\underline{\smash{{\boldmath \bfseries The Euclidean algorithm for the gcd}}}

Let us compute $\gcd(45, 66)$.
Here $66 = 1 \cdot 45 + 21$ where $q = 1$ and $r = 21$, so $\gcd(45, 66) = \gcd(21, 45)$.
Next, $45 = 2 \cdot 21 + 3$, so $\gcd(21, 45) = \gcd(3, 21)$.
Next, $21 = 7 \cdot 3 + 0$, so $\gcd(3, 21) = \gcd(0, 3) = 3$.

\begin{thm}
  Every positive integer can be written as a product of (finitely many) prime numbers $n = \prod_{i = 1} = p_1 \cdot \cdots \cdot p_r$, where $p_1, \dots, p_r$ are prime and $r \geq 0$.
\end{thm}

\begin{note}
  By convention, $1$ is the product of $0$ prime numbers.
\end{note}

\begin{prf}
  We use induction on $n \in \mathbb Z^+$.

  The theorem is true for $n = 1$.

  Suppose that $n > 1$ and that the theorem holds for smaller positive integers.
  If $n$ is prime, we are done.
  Otherwise, there is an integer $m$, $1 < m < n$, with $m \mid n$.
  Then both $m$ and $\frac{n}{m}$ are positive integers $< n$.
  So they are both products of primes.
  So $n = m \left ( \frac{n}{m} \right )$ is a product of primes.
\end{prf}

\begin{lem}
  If a prime number $p$ divides the product $m n$ of integers, then $p \mid m$ or $p \mid n$.
\end{lem}

\begin{prf}
  Suppose that $p \mid m n$ and $p \nmid m$.
  We want to show that $p \mid n$.

  Since $p \nmid m$, $\gcd(p, m) = 1$.
  So by Euclidean algorithm, we write $1 = p u + m v$ for some integers $u, v$.

  We can also write $m n = p w$ for some $w \in \mathbb Z$.
  So, multiplying $1 = p u + m v$ by $n$, we have $n = n p u + m n v = p (n u + w v)$.
  So $p \mid n$.
\end{prf}

\newpage

\section{9.28 Wednesday Week 1}

\begin{thm}[Unique factorization of integers, Euclid  ]
  Every positive integer $n$ can be written \textit{uniquely} as a product of prime numbers, that is, $n = \prod_{i = 1}^r p_i$ where $p_1, \dots p_r$ are prime.
  The uniqueness is up to reordering of the $p_i$'s.
\end{thm}

\begin{prf}
  We use (from last time) if a prime number $p$ divides $m n$ (for some $m, n \in \mathbb Z$), then $p \mid m$ or $p \mid n$.
  We showed existence of a prime factorization of $n \in \mathbb Z^+$.
  
  For uniqueness: suppose $n = \prod_{i = 1}^r p_i = \prod_{i = 1}^s q_i$ with $p_i$'s and $q_i$'s all prime and $r, s \geq 0$.

  If $r = 0$, then $n = 1$.
  Then $s = 0$: a product of $\geq 1$ prime number is $\geq 2$ since each prime is $\geq 2$.

  Otherwise, $r > 0$.
  Then $p_1$ makes sense and it is prime.
  Then $p_1$ divides $n = \prod_{i = 1}^s q_i$.
  By previous result, $p_1$ must divide $q_i$ for some $1 \leq i \leq s$.
  By reordering the $q_i$'s, we can assume that $i = 1$.
  Since $q_i$ is prime and $p_1 > 1$, we must have $p_1 = q_1$.
  Then 
  \[
    p_1 \left ( \prod_{i = 2}^r p_i \right ) = q_1 \left ( \prod_{i = 2}^s q_i \right ) = p_1 \left ( \prod_{i = 2}^s q_i \right ).
  \]
  Since $p_1 \neq 0$, it follows that $\prod_{i = 2}^r p_i = \prod_{i = 2}^s q_i$.

  That finishes the proof, by induction on $r$.
\end{prf}

\underline{\smash{{\boldmath \bfseries Equivalence relations}}}

\begin{defn}
  The {\boldmath \bfseries product} of two sets $A$ and $B$, $A \times B$, is the set of ordered pairs $(a, b)$ where $a \in A$ and $b \in B$.
  Here $(a_1, b_1) = (a_2, b_2)$ iff $a_1 = a_2$ and $b_1 = b_2$.

  $\left | A \times B \right | = \left | A \right | \left | B \right |$ if $A, B$ are finite sets.
\end{defn}

\begin{defn}
  A {\boldmath \bfseries relation} of a set $A$ with a set $B$ is a subset $R \subseteq A \times B$.
  We write $a R b$ to mean that $(a, b) \in R$.
\end{defn}

\begin{eg}
  A function $f \colon A \to B$ determines a relation, the {\boldmath \bfseries graph} $R = \left \{ (a, f(a)) : a \in A \right \}$.
\end{eg}

\begin{defn}
  An {\boldmath \bfseries equivalence relation} on a set $A$ is a relation $R \subseteq A \times A$ such that it is
  \begin{enumerate}
    \item reflexive ($\forall \, a \in A: a R a$), 

    \item symmetric ($\forall \, a, b \in A: a R b \Rightarrow b R a$), and 

    \item transitive ($\forall \, a, b, c \in A: a R b \wedge b R c \Rightarrow a R c$).
  \end{enumerate}
\end{defn}

\begin{eg}
  For any set $A$, {\boldmath \bfseries equality} is an equivalence relation on $A$.
\end{eg}

\begin{eg}
  Triangles in $\mathbb R^2$ under {\boldmath \bfseries congruence} (studied by Euclid): we say that a triangle $a$ is ``congruent'' to triangle $b$ if there is an {\boldmath \bfseries isometry} $f \colon \mathbb R^2 \to \mathbb R^2$ that maps $a$ to $b$.
\end{eg}

\begin{eg}
  The relation on $\mathbb Z \times \left \{ \mathbb Z \setminus \left \{ 0 \right \} \right \}$ given by $(a, b) \sim (c, d)$ if $a d = b c$.
  In fact this relation is equivalent to $\frac{a}{b} = \frac{c}{d} \in \mathbb Q$.
  This equivalence relation ``ensures'' that $\frac{1}{3} = \frac{2}{6} = \frac{3}{9} = \cdots$.
  It is a way to constructing $\mathbb Q$ from $\mathbb Z$.
\end{eg}

\begin{defn}
  Let $\sim$ be an equivalence relation on a set $A$.
  For each element $a$ let {\boldmath \bfseries $\overline a$} or {\boldmath \bfseries $[a]$}, the {\boldmath \bfseries equivalence class of $a$}, be the set $\left \{ b \in A : a \sim b \right \} (\subseteq A)$.

  Let $\overline A$ be the set of subsets of $A$ of the form $\overline a$ for some $a \in A$.
  $\overline A$, or $A/\sim$, is called the set of {\boldmath \bfseries equivalence classes for $\sim$}.

  Define a function $f \colon A \to \overline A$ (depending on $\sim$) by $f(a) = \overline a \in \overline A$.
  This is the {\boldmath \bfseries natural} or {\boldmath \bfseries canonical surjection} associated to $\sim$.
\end{defn}

\begin{eg}
  For the relation from Example~\ref{eg:3.8} on $\mathbb Z \times (\mathbb Z \setminus \left \{ 0 \right \})$ $A$, we can define $\mathbb Q = A/\sim$.
\end{eg}

\begin{eg}
  Define an equivalence relation on $\mathbb Z$ by $a \sim b$ if $a - b$ is even.
  Some equivalence classes are 
  \begin{align*}
    \overline 0 &= \left \{ b \in \mathbb Z : 0 \sim b \right \} = \left \{ \dots, -4, -2, 0, 2, 4, \dots \right \} \\ 
    \overline 1 &= \left \{ \dots, -3, -1, 1, 3, \dots \right \} \\ 
    \overline 5 &= \overline 1.
  \end{align*}
\end{eg}

\begin{defn}
  $\mathbb Z/2 := \mathbb Z/\sim$ for the relation in Example~\ref{eg:3.11}.
  Note that this set has exactly 2 elements.
\end{defn}

\begin{prop}
  Let $\sim$ be an equivalence relation on a set $A$.
  Then $A = \bigsqcup_{u \in \overline A} u$.
\end{prop}

\begin{prf}
  First we show that $A = \bigcup_{u \in \overline A} u$.
  For each $u \in \overline A$, $u$ is a subset of $\overline A$.
  Then $\bigcup_{u \in \overline A} u \subseteq A$.
  Conversely let $a \in A$.
  Then $a \in \overline a$ by reflexivity of $\sim$.
  So $A = \bigcup_{u \in \overline A} u$.

  Next we show that given $u, b \in \overline A$, if $u \neq v$ then $u \cap v = \varnothing$.
  Equivalently, we show that if $u, v \in \overline A$ and $u \cap v \neq \varnothing$ then $u = v$.
  The assumption means that there is an element $a \in A$ such that $a \in u$ and $a \in v$.
  By definition of $\overline A$, $u = \overline b$ and $v = \overline c$ for some $b, c \in A$.
  Since $a \in u = \overline b$ and $a \in v = \overline c$, $b \sim a$ and $c \sim a$.
  By symmetry and transitivity, $b \sim a \sim c \Rightarrow b \sim c$.
  
  To show that $\overline b = \overline c$, pick any element $e \in \overline b$, that is, $b \sim e$, $c \sim b \sim e$ so $c \sim e$.
  Then $e \in \overline c$.
  The same proof shows that any element in $\overline c$ is also in $\overline b$.
  Then $\overline b = u = v = \overline c$.
\end{prf}

\end{document}
