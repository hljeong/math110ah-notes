\documentclass{notes}

\class{MATH 110AH (Algebra)}

\notusingsubsection

\begin{document}

\section{9.23 Friday Week 0}

Groups were invented by Évariste Galois circa 1830 to discuss symmetries in mathematical objects.
For instance, the group $G$ of ``symmetries of equilateral triangles'' contains a total of $3 \cdot 2 = 6$ elements consisting of rotations and reflections.

\begin{thm}
  Finite simple groups are classified.
\end{thm}

\underline{\smash{{\boldmath \bfseries Proving things about the integers}}}

Try to use ``simple'' facts about the integers to prove complicated ones.
Recall the number systems: 
\begin{itemize}
  \item the integers $\mathbb Z = \left \{ \dots, -3, -2, -1, 0, 1, 2, 3, \dots \right \}$, 

  \item the rational numbers $\mathbb Q = \left \{ \frac{a}{b} : a, b \in \mathbb Z, b \neq 0 \right \}$, and 

  \item the real numbers $\mathbb R$ contains $\mathbb Q, \sqrt 2, \pi, \dots$, ``all the points on the line.''
\end{itemize}

\begin{defn}
  A field $\mathbb F$ (e.g. $\mathbb Q, \mathbb R, \mathbb C$, not $\mathbb Z$) is a set with elements $0, 1 \in \mathbb F$, $0 \neq 1$, and operators $+$ (addition) and $\cdot$ (multiplication), where for all $x, y \in \mathbb F$, $x + y, x y \in \mathbb F$, such that
  \begin{enumerate}
    \item $+$ is associative, commutative, $0$ is its identity, and has inverses.
    That is, 
    \begin{itemize}
      \item $\forall \, x, y, z \in \mathbb F: (x + y) + z = x + (y + z)$, 

      \item $\forall \, x, y \in \mathbb F: x + y = y + x$, 

      \item $\forall \, x \in \mathbb F: 0 + x = x$, and

      \item $\forall \, x \in \mathbb F \; \exists \, y \in \mathbb F: x + y = 0$, and writing $y = -x$.
    \end{itemize}

    \item $\cdot$ is associative, commutative, $1$ is its identity, and nonzeros have inverse.
    That is, 
    \begin{itemize}
      \item $\forall \, x, y, z \in \mathbb F: (x y) z = x (y z)$, 

      \item $\forall \, x, y \in \mathbb F: x y = y x$, 

      \item $\forall \, x \in \mathbb F: 1 \cdot x = x$, and 

      \item $\forall \, x \in \mathbb F: x \neq 0 \Rightarrow \exists \, y \in \mathbb F: x y = 1$.
    \end{itemize}

    \item Distributive law: $\forall \, x, y, z \in \mathbb F: x (y + z) = x y + x z$.
  \end{enumerate}
\end{defn}

\begin{defn}
  An ordered field (for example, $\mathbb R, \mathbb Q$, not $\mathbb C$) $\mathbb F$ is a field with a given subset $P \subset \mathbb F$ called the positive elements such that 
  \begin{enumerate}
    \item for all $x \in \mathbb F$, exactly one of $x \in P$, $x = 0$, $-x \in P$ is true (we say $x \in \mathbb F$ is negative if $-x \in P$), and 

    \item for all $x, y \in P$, $x + y, x y \in P$.
  \end{enumerate}
\end{defn}

\newpage

\underline{\smash{\boldmath \bfseries How to use these axioms to prove inequalities}}

\begin{defn}
  For an ordered field $\mathbb F$, we say for $x, y \in \mathbb F$ that $x < y$ if $y - x = y + (-x)$ is positive (so $x$ is positive iff $x > 0$).

  Likewise $x \leq y$ if $y - x$ is positive or $0$.
\end{defn}

\begin{lem}
  \begin{enumerate}
    \item For any $x, y \in \mathbb R$, exactly one of $x < y$, $x = y$, $x > y$ is true.

    \item $1$ is positive.

    \item For $a, b, c \in \mathbb R$, if $a < b$ then $a + c < b + c$.

    \item If $a, b, c \in \mathbb R$, $a \geq 0$, and $b \geq c$, then $a b \geq a c$.
  \end{enumerate}
\end{lem}

\begin{prf}
  \begin{enumerate}
    \item $y - x$ is either positive, negative, or $0$.

    \item Note that $1 \neq 0$.
    Then either $1$ is positive or $-1$ is positive.
    If $-1$ is positive then $(-1)^2 = 1$ is positive, resulting in a contradiction.

    \item Note that $(b + c) - (a + c) = b - a$ is positive.

    \item Note that for all $a \in \mathbb R$ we have $0 \cdot a = 0$.
    Then 
    \begin{align*}
      a (b - c) &\geq 0 \\ 
      a b - a c &\geq 0 \\ 
      a b &\geq a c.
    \end{align*}
  \end{enumerate}
\end{prf}

\underline{\smash{{\boldmath \bfseries The big property of the integers is the inductive or well-ordering principle}}}

The integers $\mathbb Z$ are a subset of $\mathbb Q$ (or $\mathbb R$).
There are $0, 1 \in \mathbb Z$ and if $x, y \in \mathbb Z$ then $x + y, x y, -x \in \mathbb Z$.

\begin{thm}[Well-ordering principle]
  Let $\mathbb Z^+ = \left \{ x \in \mathbb Z : x > 0 \right \} = \left \{ 1, 2, 3, 4, \dots \right \}$.
  Let $S$ be a nonempty subset of $\mathbb Z^+$.
  Then $S$ contains a smallest element, that is, 
  \[
    \exists \, x \in S \; \forall \, y \in S: x \leq y.
  \]
\end{thm}

An example of what we can prove using this is: 
\begin{prop}
  There is no integer $N$ with $0 < N < 1$.
\end{prop}

\begin{prf}
  Let $S = \left \{ n \in \mathbb Z : 0 < n < 1 \right \}$.
  Let $S \neq \varnothing$, then by the well-ordering principle $S$ has a smallest element $N$.
  Since $N < 1$, $N^2 < N \cdot 1 = N$.
  Since $N^2$ is also an integer, this is a contradiction.
  Then $S = \varnothing$, that is, there is no integer $N \in (0, 1)$.
\end{prf}

\begin{thm}[Induction]
  For each $n \in \mathbb Z^+$, let $P(n)$ be a statement that could be true or false.
  Suppose $P(1)$ is true and that for any $n \in \mathbb Z^+$, if $P(n)$ is true then $P(n + 1)$ is true.
  Then $P(n)$ is true for all $n \in \mathbb Z^+$.
\end{thm}

\begin{prf}
  Let $S = \left \{ n \in \mathbb Z^+ : P(n) \text{ is false} \right \}$.
  Suppose $S \neq \varnothing$, then by the well-ordering principle $S$ has a smallest element $N$.
  Since $P(1)$ is true, $N \neq 1$.
  Then $N > 1$.
  Then $N \geq 2$ since there are no integers in $(1, 2)$.
  Then $N - 1 \in \mathbb Z^+$.
  Since $P(N)$ would be true if $P(N - 1)$ were true, $P(N - 1)$ is not true.
  Then $N - 1 \in S$, a contradiction.
  Then $S = \varnothing$.
\end{prf}

\end{document}
